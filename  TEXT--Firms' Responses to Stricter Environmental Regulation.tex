\documentclass[12pt,english]{article}
\usepackage{mathpazo}
\usepackage{helvet}
\usepackage[T1]{fontenc}
\usepackage[utf8]{inputenc}
\usepackage{array}
\usepackage{amsmath}
\usepackage[tmargin=2.54cm,bmargin=2.54cm,lmargin=2.54cm,rmargin=2.54cm]{geometry}
\usepackage{color}
\usepackage{babel}
\usepackage{float}
\usepackage{booktabs}
\usepackage{graphicx}
\usepackage{setspace}
\newtheorem{proposition}{Proposition}
\makeatletter
\providecommand{\tabularnewline}{\\}
\newcommand{\lyxdot}{.}
\usepackage{pdflscape}
\usepackage{epstopdf}
\usepackage{babel}
\usepackage{multirow}
\usepackage[justification=centering]{caption}
\usepackage{lineno}
\newcommand\fnote[1]{\captionsetup{justification=justified, textfont=footnotesize}\caption*{#1}}
%\linenumbers
\usepackage[lofdepth,lotdepth]{subfig}
\usepackage[authoryear]{natbib}
\DeclareGraphicsExtensions{.eps,.mps,.pdf,.jpg,.png}
\DeclareGraphicsRule{*}{eps}{*}{}
\setlength{\bibsep}{0.0pt}
\usepackage[dvipsnames]{xcolor}
\usepackage{hyperref}
\hypersetup{
  colorlinks=true,
  citecolor=BrickRed,
  linkcolor=BrickRed,
   urlcolor=Black
}
\makeatother
\begin{document}
\title{Firms' Responses to Stricter Environmental Regulations: to Be Light Green or Bright Green?}

\author{Haichao Fan, Joshua Graff-Zivin, Zonglai Kou, Xueyue Liu, Huanhuan Wang \footnote{Haichao Fan, Institute of World Economy, School of Economics, Fudan University, Shanghai, China. Joshua Graff-Zivin, Department of Economics and School of Global Policy and Strategy, University of California, San Diego, La Jolla, CA, USA. Zonglai Kou, China Center for Economic Studies, School of Economics, Fudan University, Shanghai, China. Xueyue Liu, China Center for Economic Studies, School of Economics, Fudan University, Shanghai, China. Huanhuan Wang, School of Law, East China Normal University, Shanghai, China. Contacts: Fan, fan\_haichao@fudan.edu.cn; Kou, zlkou@fudan.edu.cn; Graff-Zivin, jgraffzivin@ucsd.edu; Liu, xueyueliu@fudan.edu.cn; Wang, hhwang@law.ecnu.edu.cn. We acknowledge financial support from the National Natural Science Foundation of China (no. 71603155). All errors are our own.}}
\date{October 2019}
\maketitle
\begin{abstract}

This paper examines the effect of stringent environmental regulations on firms' environmental practices, economic performance, and environmental innovation. Reducing COD levels by 10\% relative to 2005 levels is an aim of the Chinese 11$^{th}$ Five-Year Plan. Using a difference-in-differences framework based on comprehensive firm-level dataset, we find that more stringent environmental regulations faced by firms are positively associated with a greater probability of reducing COD emissions; also, there exists an evident heterogeneous effect across industries with different pollution intensities. Stricter environmental regulations also account for the sharp decline in firms' profits, capital, and labor. After executing a complete chain of tests of the underlying mechanisms, we find that firms rely more on recycling and abatement investment than on innovations when meeting environmental requirements.

\end{abstract}
Key words: Environmental Regulations, Emissions Reduction Targets, Decomposition\\
JEL Classification: Q53, O31, D22, K32
\newpage
\onehalfspacing
\section{Introduction}

From 2006 to 2010, emissions of major regulated water pollutants like chemical oxygen demand (COD) from Chinese manufacturing fell by 23.1$\%$ relative to the 2005 level, even as China has boosted its manufacturing output by 137.5$\%$ in between.\footnote{COD is an indicative measure of the amount of oxygen that can be consumed by reactions in a volume of water to reflect water quality. In China, COD emissions have been rigorously regulated since 1980s.} When looking into pattern of the cleanup shown in Figure \ref{fig:COD emissions}, we find the turning point first appearing in 2006, the starting year of Chinese 11$^{th}$ Five-Year Plan characterized by stringent emissions target control and enforcement. The temporal coincidence provides us a very first hint on linking environmental improvement with environmental regulations. Have stringent environmental regulations contributed to cleanup? And, if so, to what extent? And what are the main driving forces for the effect? Are firms turning green lightly or brightly?\footnote{Despite the origins in \citet*{hicks1963theory} and \citet*{porter1991american} on induced innovation, to answer the above questions, we still need to distinguish between different types of environmental-friendly technologies. When it comes to COD emissions, the first category is end-of-pipe technologies, which directly treat wastewater to reduce COD levels. End-of-pipe technologies are called \textit{treatment technologies}. The second category of technologies adopted in freshwater saving and wastewater recycling objectively reduce COD levels. They are called \textit{recycling technologies}. The last category, called \textit{green innovations}, improves production techniques and processes to reduce COD levels at the source. When faced with various green technologies, firms might adopt \textit{treatment technologies} and \textit{recycling technologies} to reduce pollution. Here, we refer to this approach as "light green". Or they might innovate, especially pertaining to \textit{green innovations}, through green patent applications. This pattern is informative about firms' inherent innovation strength and generally followed by better environmental performance. Here, we refer to this approach as "bright green", following the environmental politics literature. Whether firms are greening lightly or brightly is critical in understanding the whole scenario about driving forces, and this knowledge further affects firms' improved environmental performance.}

\begin{figure}[htbp]
	\caption{Chemical Oxygen Demand Emissions in Chinese Manufacturing}
	\label{fig:COD emissions}
	\centering
	\includegraphics[scale=0.8]{Facts-COD.eps}
\end{figure}

In this paper, we study the effects of stringent environmental regulations on firms' pollution emissions and their related economic performance, and, more importantly, we also study the inherent driving forces behind firms' responses. Even though historically established from the 9$^{th}$ Five-Year Plan, emissions target control in China hasn't been seriously treated until the 11$^{th}$ Five-Year Plan. In view of failure in pollution abatement during the 9$^{th}$ and 10$^{th}$ Five-Year Plan, China began to strengthen emissions reduction target schemes by subdividing mandatory national emissions target among all levels of governments and in the end, polluters. Accordingly, substantive enhancement on legal enforcement has been made by imposing comprehensive legal liabilities, such as governmental officers' achievement evaluation, accountability, and promotion, as well as stricter administrative penalties for polluters. The large variation of regulation stringency among cities due to disparate emissions control targets set from central to local government is beneficial to identify causal effects between environmental regulations and firms' response, and how it varied across industries. Moreover, facilitated by mandatory COD emissions reduction target scheme substantively reinforced by the 11$^{th}$ Five-Year Plan since 2006, we mainly rely on difference-in-differences (DID) strategy to study the effect of environmental regulations before and after 2006.

To begin, we document three stylized facts pertaining to COD emissions from manufacturing in China through a set of analyses. First, faced with different environmental regulation stringency, firms in tightly regulated regions reduce more pollutants than those in loosely regulated regions. Second, the effect of environmental regulation stringency on firms' pollution emissions varied across industries with different polluting intensity. Industrial polluting intensity positively reinforces the effect of environmental regulation stringency on firms' pollution reduction. Third, "composition effect" adjustment of market share of each industry and "technique effect" lowered pollution intensity brought by technique progress are both responsible for decline of water pollutants from Chinese manufacturing upon stricter environmental regulations, whereas the "technique effect" is the predominant causal factor accounting for environmental improvement.

We construct a rich firm-level panel dataset drawing from multiple sources: firm-level pollution data from Annual Environmental Survey of Polluting Firms (AESPF), firm-level economic data from Annual Survey of Industrial Firms (ASIF), firms' green patent data from Patent Dataset, maintained by China National Intellectual Property Administration (CNIPA), firms' environmental penalty data from Institute of Public and Environmental Affairs (IPEA), city-level data from statistical yearbook. This dataset contains rich information on firm-level and city-level variables, including environmental regulation stringency and firms' environment-related activities, economic performance, patent, and legal enforcement.

We then employ a DID strategy to study the effect of environmental regulation stringency. We find that, more stringent environmental regulations faced by firms are positively associated greater probability of reducing COD emissions after 2006. For example, among all the large cities, the COD emissions reduction magnitude in Shanghai, who ranks first in environmental regulation stringency concerning COD emissions reduction target, is nearly 30$\%$ larger than that of Kunming, the bottom one with the least control target. A with-in firm decomposition further shows that, pertaining to changes in pollution reduction across cities with different environmental regulation stringency after 2006, around 30$\%$ of them can be explained by within-firm "scale effect", that is, drop in firms' total output, whereas 70$\%$ of them can be explained by within-firm "technique effect", including but not limited to adoption of pollution abatement facilities, introduction of cleaner production process and recycling use of energy inputs. Our baseline results are further proved by a differential time trends test and robustness check on other pollutants.

Considering large variation in COD pollution intensity across industries, we resort to study the heterogeneous effects, to verify whether industrial polluting intensity positively reinforces the effect of environmental regulation stringency on firms' pollution reduction. We find that, firms which belong to heavily polluting industries located in cities with more stringent environmental regulations cut down much more pollutants, compared with their intra-city counterparts in less-polluting industries. Much more sensibility of firms’ responses in heavily polluting industries to more stringent environmental regulations lends solid evidence to our stylized facts about the inter-industry allocation.

To find out whether more stringent environmental regulations have stimulated firms to become light green or bright green, we execute a broad spectrum of tests about roles of recycling practice (recycling of wastewater), adoption of pollution abatement facilities (waste water treatment), and firms’ technology progress (green patent), among others. First, we reassure the tightened legal enforcement on emissions target control explicated by the positive effect of environmental regulations on the probability that firms were punished after 2006. Second, we employ a finer within-firm decomposition by introducing another pollutant, effluent, which is industrial wastewater discharged into environment. We find that, the effect of environmental regulation stringency on COD emissions reduction is mostly attributable to decreased discharge of effluent, and the decline is much more declared for heavily polluting industries. Third, we turn to examine the adjustment of firm’s total water consumption and freshwater consumption per unit of output affected by environmental regulations. Fourth, according to our tests, the increasing of control devices and their treatment capacity is more claimed for firms in cities with stringent environmental regulations than their counterparts in cities with weaken environmental regulations after 2006. Relative to firms in cleaner industries, firms in dirtier industries tend to be more progressive in expanding pollution treatment abilities. Last, we focus on firm’s performance in green innovation induced by stricter environmental regulations. However, little evidence that strict environmental regulations lead to an increase in green patents and water-related green patent applications is found, regardless of what industry firms belong to. Therefore, we are able to infer that, emissions target control, at least during the 11$^{th}$ Five-Year Plan, failed to stimulate firms' innovation (called "bright green"), maybe because other lower-cost countermeasures, such as pollution abatement facilities (called "light green"), are sufficient to meet the target. How to stimulate firms to shift from adoption of end-of-pipe treatment technology (light green) to innovations (bright green) still needs further investigation.

After verifying the positive effect of environmental regulations on firms’ pollution reduction, especially for those in heavily polluted industries, we turn to examine firms’ economic responses. Significantly reduced pollution, output, and pollution per unit are believed to influence firms’ economic performance. To this end, we accordingly conduct a test to further assess the impact of environmental regulations on firms' economic performance including profits, capital, labor and market share. The empirical results show that these four indicators unexceptionally decrease across all industries. Also, as for firms belonging to heavily polluting industries located in cities with more stringent environmental regulations, their profits, capital, labor and market share decline more compared with their counterparts in less-polluting industries located in cities with weaken regulations. A decrease in output, input and market share of firms might shed light on potential adjustment of firm relocation. To provide more evidence on somewhat "internal" variant of the pollution havens hypothesis, we therefore further testify the impact of environmental regulation on firm entry.

There already exists an extensive literature on the effects of environmental regulations on emissions reduction in manufacturing activity and environmental quality improvement \citep*{nelson1993differential,chay2005does,greenstone2014environmental}.\footnote{Despite the emerging but still limited research using firm-level pollution data \citep*{martin2011energy,zhang2018does}, most studies rely on macro-level data. We add to the former literature by using comprehensive firm-level pollution and economic data to provide micro-level evidence on whether and to what extent changes in regulatory stringency have contributed to not only firms' emissions reduction but also their economic loss. More importantly, by integrating industrial variance in pollution intensity, we also find notable cross-industry heterogeneity in firms' environmental performance.} Besides, researchers are also curious about mechanisms underlying regulation-induced environmental cleanup. As \citet*{levinson2009technology}, we prove that the technique effect plays the predominant role in the environmental improvement. The way how technology matters is closely related to literature on environmental-friendly innovation (such as green patent) induced by environmental protection. Some of the researchers find positive effects \citep*{acemoglu2016transition,aghion2016carbon,gutierrez2018abatement,aghion2019environmental}, while some others find negative ones \citep*{nelson1993differential,gray1998environmental,gans2012innovation}. As \citet*{gutierrez2018abatement} theoretically proves that tighter climate policy including emissions cap does not necessarily improve innovations, we find little evidence on firms' green patent application in reaction to stringent environmental regulations. Our research is informative to this important issue by further identifying firms' adoption of other environmental-friendly technologies, such as abatement facilities to reduce COD level in effluent, freshwater saving, and wastewater recycling technologies, objectively reduce COD levels.

This research is also linked to literature on effect of environmental regulations on microeconomic activities of regulated firms and industries, on their employment \citep*{henderson1996effects,greenstone2002impacts,walker2013transitional}, firm productivity \citep*{berman2001environmental,greenstone2012effects}, industrial location \citep*{henderson1996effects,becker2000effects,chen2018consequences}, trade-environment links \citep*{gutierrez2018abatement}, and on export and foreign direct investment \citep*{keller2002pollution,cai2016does,shi2018environmental}, among others. Our paper is also related to studies executing decomposition of change of emissions reduction that can be explained by various effects \citep*{levinson2009technology,martin2011energy,shapiro2018pollution}. We complement the literature by not only implementing intra-industry decomposition to find out the contribution of manufacturing scale, composition of industries and technology, but also decomposing within-firm behaviors to more clearly portray the role of within-firm scale effect and within-firm technique effect. %Some others evaluate how environmental regulation influence emissions reduction in manufacturing activity and induce improvements in environmental quality (Chay and Greenstone 2003, Nelson et al. 1993; Greenstone and Hanna, 2014). We add to this literature by using comprehensive firm level pollution and economic data to provide for micro-level evidences, on whether and to what extent changes in regulatory stringency have contributed to firm's emissions reduction and economic loss. More importantly, by integrating industrial variance in pollution intensity, we also find notable cross industry heterogeneity in firms' environmental performance. Different from emerging but still limited researches on firm's responses to environmental regulation such as (Martin, 2011; Zhang et al. 2018; Chen et al., 2018), we build a chain to systematically link firm's environmental performance and business performance to investigate the underlying mechanisms for firm's specific reactions to stricter regulations.

%Our research is also linked to a strand of literature on technology upgrading induced by environmental protection. As for whether and the degree to which caring for better environment have contributed to transition to cleaner technology, some of the researches, relatively old although, find negative effects (Nelson, Tietenberg and Donihue, 1993, Gray and Shadbegian, 1998), whereas some others find positive effects (Aghion et al., 2016; Acemoglu et al., 2016; Gutierrez and Teshima, 2018; Aghion et al., 2019). Our results are informative to the literature that we define firms innovation as "green patent", which is more sharply linked with firm's environment-related behaviors. In addition, as Gutierrez and Teshima (2018), which is closely related to our research, proves, firms might lower their emissions via cost-saving process improvements, either through technology upgrading or management practice, but with investment in abatement. Even though we don't find augment of firm's green patents, we, quite on the contrary, do find clear evidences on firm's adoption of clean recycling technology on water saving as well as pollution abatement facilities in an endeavor to meet emissions targets. Despite of common acceptance on superiority of market-based over comman-and-control instruments in terms of environmental regulatory induced innovation (Ambec et al., 2013), our finding implies that it is not the type of regulation matters, but whether or not it exerts continuous pressure on firms.\footnote{As an extension to empirical results about impacts on firm's green patent in \ref{sec:mechanism}, we use COD emissions target control in the 12$^{th}$ Five-year Plan from 2011 to 2015 which requires a further 10 $\%$ reduction in COD emissions lower than 2010 level as observing objects. We find increase of firm's green patent associated with much stricter emissions control. Results are available upon request.}
%Gutierrez and Teshima (2018) prove that import competition induced plants in Mexico to increase energy efficiency, reduce emissions, and in turn reduce direct investment in environmental protection. Their finding suggest that the general technology upgrading effect of any policy could be an important determinant of environmental performance in developing countries and that this effect may not be captured in abatement data. Aghion et al. (2016) find a sizable impact of carbon taxs on the direction of innovation in the automobile industry and further provide evidence that clean innovation has a self-perpetuating nature feeding on its past success. By developing an endogeneous growth model in which clean and dirty technologies compete in production, Acemoglu et al. (2016) show that if dirty technologies are more advanced, the transition to clean techonology can be difficult. Carbon taxs and research subsidies may encourage production and innovation in clean technologies, though the transition will typically slow. Acemoglu et al. (2012) show that a combination of research subsidies and carbon taxes can successfully redirect technological change toward cleaner technologies. Aghion et al. (2019) find that pro-environmental attitudes and competition have both a significant positive effect on the probability for a firm to patent more in the clean direction and the interaction term between attitude and competition is also significantly positive.
%The idea that regulation can spur technological innovation is based on the concept of induced innovation, which goes all the way back to Hicks (1932).
%Environmental regulation and productivity: Berman and Bui (2001) find that abatement cost measures may grossly overstate the economic cost of environmental regulation as abatement can increase productivity. Greenstone, List and Syverson (2012) find that stricter air quality regulations are associatd with 2.6 percent decline in TFP. Greenstone (2002) finds that greater regulatory oversight results in lost in jobs, capital stock and output in pollution-intensive industries.

%As a core solution facing the world economy when securing better environment, successful transition to clean technology has attracted broad attention of economic researchers. Our paper also contributes to literature on transition from dirty to clean technology when.  Porter hypothesis (Porter 1991): well-designed regulation could actually increase competitiveness "strict environmental regulations do not inevitably hinder competitive advantage against rivals; indeed, they often enhance it." [Until that time, the traditional view of environmental regulation, held by virtually all economists, was that requiring firms to reduce an externality like pollution necessarily restricted their options and thus by definition reduces their profits. After all, if profitable oppotunities existed to reduce pollution, profit-maximizing firms would already be taking advantage of them]  innovations may take several years to develop. examine the dynamic nature of the relationshio between regulation, innovation, productivity and competitiveness. The effect of environmental regulation The emissions reduction control, or at least the target set to the extent in China's 11$^{th}$ Five-year Plan, has successfully induce incremental innovations of firms.

The rest of the paper is organized as follows: Section \ref{sec:institution} discusses relevant institutional background and presents stylized facts based on preliminary data analysis. Section \ref{sec:empirical,data and measurement} provides the main empirical specification and describes the data. The main empirical results are presented in Section \ref{sec:results}, followed by discussion on mechanism underlying the effect of environmental regulations in Section \ref{sec:mechanism}. In Section \ref{sec:performance}, we discuss firms' other economic performance upon stricter environmental regulations. Section \ref{sec:conclusion} concludes.

\section{Institutional Background and Stylized Facts}\label{sec:institution}

\subsection{Institutional Background}\label{sec:background}

Among all regulatory tools for emissions reduction, concentration control and emissions target control are two of the fundamental ones. The former one aims to require pollution sources (e.g., industrial facilities) to control for concentration of contaminants to comply with national pollution control standards, while the later sets compulsory emissions cap as well as reduction targets of regulated pollutants. Chinese environmental regulation features coexistence of the two schemes, since prevalent adoption of emissions target control across the nation from the 9$^{th}$ Five-Year Plan during 1996 to 2000. To account for invalidity of concentration control in limiting total scale of pollutants entering environment, emissions target control in China mainly focus on "critical pollutants" by setting national reduction targets followed by top-down subdivision from the central government to provinces then to cities. %Polluters are commonly issued with a permit to specify their emissions cap and reduction targets. The target is so mandatory that all levels are obliged to fulfill the emissions targets assigned with, otherwise they will face various punishment for act of defiance.

Unfortunately, the emissions reduction goal wasn't accomplished during the 9$^{th}$ Five-Year Plan period and the term followed. Among all the targeting indicators set up in the 10$^{th}$ Five-Year Plan, pollution control targets are the only ones unrealized. COD emissions in 2005 is merely 2$\%$ lower than that of baseline year 2000. To reverse the failure in accomplishing emissions control targets, the government substantively strengthened the emissions target control scheme in the 11$^{th}$ Five-Year Plan period from 2006 to 2010. After decomposing the nation goal to provinces, goal statement on emissions reduction are accordingly signed between each provincial government and the national Ministry of Environmental Protection. The performance of governmental officers on fulfilling the duties and emissions mandates, according to Measures on Accomplishment Evaluation of Critical Pollutants Emissions Control Target, will be evaluated and incorporated into their competency assessment, exerting potential impact on their accountability and promotion. Moreover, other complementary regulations, such as Statistical Measures on Critical Pollutants Emissions Control Target and Interim Verifying Measures on Critical Pollutants Emissions Control Target, were enacted in 2006. As a result, the 10$\%$ reduction target of two "critical pollutants"---COD and SO$_{2}$---lower than 2005 level has been excessively achieved.\footnote{The scope of critical pollutants in each five-year plan varied. The 9$^{th}$ Five-Year Plan defined 12 pollutants as "critical pollutants." The scope was narrowed into 6 pollutants during the 10$^{th}$ Five-Year Plan. COD and SO$_{2}$ were the only two focuses of the 11$^{th}$ Five-Year Plan.} We therefore infer that the 11$^{th}$ Five-Year Plan period is the turning point pertaining to effectiveness of emissions target control. %The framework of emissions target control was built up since then by the way of three main subdivision of the emissions target---from the central government to provinces, from provinces to cities and counties, and from government to polluters. when 6 pollutants including COD was designated as "critical pollutants".  Bearing in mind the lesson of the past

\subsection{Stylized Facts}\label{sec:facts}

As core of the emissions control target scheme, provinces are assigned with different emissions reduction target within the 11$^{th}$ Five-Year Plan. For instance, Guangdong with the highest target is obliged to reduce 0.159 million tons of COD during 2006 and 2010, while Xizang, Qinghai, and Xinjiang at the bottom have no duty on COD emissions control. Thriving to achieve discrete mandatory targets, provinces accordingly execute environmental regulations with great discrepancy in stringency. Bearing how legal enforcement varies across regions in mind, we plot COD emissions changes for firms located in different provinces, to provide hint on whether COD emissions level is associated with stringency of environmental regulations. Provinces are accordingly divided into two groups---tightly regulated and loosely regulated provinces---based on whether their COD reduction targets is above or below the median target mandated in the 11$^{th}$ Five-Year Plan. We sum the volume of firms' COD emissions in each group, and Figure \ref{fig:COD provinces} presents the results. The blue line, associated with the left \emph{y}-axis, plots the overall COD emissions of tightly regulated provinces, whereas the red dashed line corresponding to the right \emph{y}-axis is indicative of COD emissions of the loosely regulated group. It is interesting to note that, according to Figure \ref{fig:COD provinces}, COD emissions in tightly regulated provinces decreased, then reveals a sharp plunge from 2007 until reach a historically low level in 2010. COD emissions in loosely regulated provinces, quite on the contrary, is largely stable with slight decline between 2005 and 2010. We summarize the first stylized fact as follows:

\textit{Stylized fact 1. In response to different environmental regulation stringency, firms in tightly regulated regions reduce more pollutants than those in loosely regulated regions.}

\begin{figure}[htbp]
	\caption{COD Emissions in Manufacturing in Tightly and Loosely Regulated Provinces}
	\label{fig:COD provinces}
	\centering
	\includegraphics[scale=0.8]{Facts-COD-prov.eps}
\end{figure}

To further explore whether the effect of environmental regulation stringency on firms' pollution reduction depends on pollution intensity difference across industries, we take industrial pollution intensity into consideration. After all, heavily polluting industries contribute most of the pollutants and often viewed as main sources of emissions reduction. Apart from separation between highly and loosely regulated provinces, we divide all 30 manufacturing industries at CIC-2 (Chinese Industry Classification) level into heavily polluting industries and lightly polluting industries on the ground that whether their pollution intensity is above or below the median level of all manufacturing industries. As demonstrated by graph at the top of Figure \ref{fig:COD industries}, no matter located in tightly or loosely regulated provinces, firms belonging to heavily polluting industries both experience decline in COD emissions. The graph at the bottom which plots changes of firms' COD emissions in lightly polluting industries, however, shows that COD emissions of firms located in tightly regulated provinces even surprisingly increases (the blue-dotted line) during the 11$^{th}$ Five-Year Plan period. Thus, we come into the second stylized fact:

\textit{Stylized fact 2. The effect of environmental regulation stringency on firms' COD emissions varied across industries with different polluting intensity. Industrial polluting intensity positively reinforce the effect of environmental regulation stringency on firms' pollution reduction.}

\begin{figure}[htbp]
	\caption{COD Emissions in Manufacturing with Different Pollution Intensities in Tightly and Loosely Regulated Provinces }
	\label{fig:COD industries}
	\centering
	\includegraphics[scale=0.1]{Facts-COD-ind.png}
\end{figure}


\subsection {A Statistical Decomposition of China's COD Emissions, 2001--2010}\label{sec:decomposition}

Probing for cause of COD emissions change in a much broader spectrum, we, in this subsection, decompose changes in total manufacturing COD emissions into changes that can be explained by total scale of manufacturing output, the composition of products produced, and the pollution intensity of a given set of products following \citet*{levinson2009technology}. Total pollution $P$ from manufacturing, equals to sum of pollution from each industry $p_{s}$, which can be further written as sum of output of each industry, $y_{s}$ multiplied by $e_{s}$, that is pollution intensity of that industry denoted by amount of pollution per unit of output value. Alternatively, we can also write manufacturing pollution as equal to total output $Y$ times each industry's share of total output ($v_{s}=y_{s}/Y$), multiplied by $e_{s}$. The equation is as follows:

\begin{equation} \label{eq:decompostion}
P=\sum_{s} p_{s}=\sum_{s}y_{s}e_{s}=Y\sum_{s}v_{s}e_{s}.
\end{equation}

In vector notation, we have

\begin{equation}\label{eq:decompostion vector}
P=Yv'e,
\end{equation}
where $v'$ and $e$ are vectors representing market share of each of the $n$ industries and their pollution intensity, respectively.

Differentiating Equation \ref{eq:decompostion vector} totally, we obtain

\begin{equation}\label{eq:decompostion differentiation}
dP=v'edY+Yedv'+Yv'de.
\end{equation}

The first term on the right-hand side of Equation \ref{eq:decompostion differentiation} is the scale effect, indicating changes of total pollution that can be explained by increase of overall scale of manufacturing, holding the composition of industries and industrial pollution intensity fixed. The second term is composition effect, revealing the change of industries mix, holding manufacturing scale and industrial pollution intensity fixed. The third term is technique effect, which account for changes in pollution intensities of each industry, holding scale and composition unchanged.

%We use Chinese firm level pollution data and output data from 2001 to 2010 to execute our decomposition. \footnote{For more details on data description, please refer to posterior Section \ref{sec:data}.}
Figure \ref{fig:decomposition} illustrate the resulting statistical decomposition for COD emissions in China. The top red solid line depicts COD emissions that would has been if let market share of each industries and its pollution intensity remain fixed at 2001 level but the overall manufacturing output had equaled to obeserved historical values. The middle blue dashed line in Figure \ref{fig:decomposition} plots the change of COD emissions if we keep amount of pollution per unit of output value, that is pollution intensity of each industry as in 2001 level, but the overall manufacturing output and market share of each industry equals to their obeserved historical values. It reflects comprehensive impact of overall manufacturing scale and composition of industries on COD emissions. The bottom green dashed line plots actural COD emissions from manufacturing, explaining overall impact from manufacturing scale, composition of industries and technology.

Figure \ref{fig:decomposition} carries rich information on driving force of firms COD emissions. First, the gap between the red solid line and the blue dashed line shows change of COD emissions that can be explained by change in composition of industries in manufacturing. The increased gap between the two provide clear evidence that composition between manufacturing products that require high and low amounts of pollution emissions for production has changed over time. %, with a much more stark change happened when more stringent environmental regulation represented by more effective emissions reduction target control introduced since 2006.
 The change, implicitly though it is, shows the shrink of dirtier industries but the expansion of cleaner industries in Chinese manufacturing.
 Second, the gap between the blue-dashed line and green-dashed line shows how much pollution intensity at industrial level decline is account for COD emissions change from manufacturing in China. Combining with the gap between the solid-red line and the blue-dashed line, we can conclude that around 18.97$\%$ of COD emissions change can be explained by "composition effect," whereas the "technique effect" accounts for 82.03$\%$ of COD emissions change. Thus, our third stylized fact is as follows.

\textit{Stylized Fact 3. "Composition effect" (adjustment of market share of each industry) and "technique effect" (lowered pollution intensity brought by technique progress) are both responsible for decline of COD emissions from Chinese manufacturing upon stricter environmental regulations. The environmental improvement is, however, mostly brought by "technique effect".}

\begin{figure}[htbp]
	\caption{Decomposition of the Three Main Effects Causing COD Emissions in Chinese Manufacturing}
	\label{fig:decomposition}
	\centering
	\includegraphics[scale=0.8]{decomposition.eps}
\end{figure}

\section {Empirical Specification and Data Description}\label{sec:empirical,data and measurement}

\subsection{Empirical Specification}\label{sec:specification}

The three previous stylized facts provide intuitive evidences that strengthening emissions reduction control enforcement and industrial pollution intensity have positive effects on firms' pollution control. The large variation of regulation stringency among cities due to disparate emissions control targets set from central to local government is beneficial to identify causal effects between environmental regulations and firms' responses, and how it varied across industries.
%  emissions target control from 2006 to 2010 provides an excellent opportunity to identify the causal effect of environmental regulation stringency on firms' pollution reduction. First, as national mandatory scheme, emissions target control is superior to regional policies such as "Two-controlled-zones" when assessing effectiveness of national environmental regulation. The time limit of the Five-year plan ideally help to rule out temporal-correlated confounding factors. Second, to fulfill national reduction targets, provinces, cities, counties and polluters has been allocated with disparate targets, which are also mandatory in legal force. Thus, firms located in different cities (the basic level of our empirical analysis) are exposed to environmental regulation with varied stringency. The variance facilitate our investigation on firms' response to environmental regulation. Last but not least, equipped with a serial of complementary regulations,
We therefore adopt a DID strategy facilitated by mandatory COD emissions reduction target scheme substantively strengthened by the 11$^{th}$ Five-Year Plan from 2006. In fear of potential governmental accountabilities and legal liabilities, cities assigned with high reduction targets have accordingly shifted into a stricter environmental regulation pattern. We compare firms' pollution in cities with more stringent environmental regulations before and after 2006 with the equivalent changes in cities with less stringent environmental regulations based on the following specification:
\begin{equation} \label{eq:baseline_reg}
y_{it}=\beta_1 \text{R}_{c} \times \text{Post}_{t} + \gamma\text{Z}_{c,t-1} +\varphi_{i}+\varphi_{t} +\epsilon_{it},
\end{equation}
where the dependent variable, $y_{it}$, refers to firm $i$'s pollution-related activities, including log value of COD, output, and pollution intensity at year $t$.\footnote{Firms could possibly respond to stricter mandated emissions targets by producing less or lowering the pollution intensity of emissions per output through, for example, using less polluting input, recycling usage of inputs, adopting pollution abatement devices or green innovation.} ${R}_{c} $ is a measure of environmental regulation stringency denoted by total COD reduction target mandated by the 11$^{th}$ Five-Year Plan for city $c$ from 2006 to 2010. ${Post}_{t} $ is a dummy variable equals to 0 for all years before 2006, and to 1 from 2006 and years onward. $Z_{c, t-1} $is a vector of city-level characteristics including log gross domestic product (GDP) per capita and log population at year $t-1$. $\varphi_i$ is firm fixed effects accounting for unobserved time-invariant differences across firms that may affect firms' polluting activities. In other words, we focus on the within-firm variation arising from changes in environmental regulation stringency faced by the firm. $\varphi_{t}$ is year fixed effects capturing common economic factors affecting all the cities. $\epsilon_{it}$ is the standard errors clustered at city level capturing all unobserved factors that influence $y_{it}$.

Intangible though it is, environmental regulation stringency could reasonably be proxied by different setting of COD emissions reduction targets mandated by the $11^{th}$ Five-Year Plans. Considering that open-accessed official document only provide emissions reduction targets at provincial level, we follow \citet*{chen2018consequences} to construct emissions reduction targets at the city level, that is, ${R}_{c}$, in Equation \ref{eq:baseline_reg}, as follows:
\begin{equation} \label{eq:${R}_{c}$}
\Delta COD_{c,05-10}=\Delta COD_{p,05-10} \times\sum_{i=1}^{39} u_{i}\dfrac{\text{output value of industry $i$ in city $c$}}{\text{output value of industry $i$ in province $p$}},
\end{equation}
where $ \Delta COD_{c,05-10} $ is COD emissions reduction targets in the $11_{th}$ Five-Year commitment period for city $c$. The second term on the right-hand side of the equation is a measure of city's proportion to province's total output value across all the 39 two-digit industries, weighted by each industry’s proportion of COD emissions to total COD emissions from manufacturers, $u_{i}$.\footnote{During the commitment period of the 11$^{th}$ Five-Year Plan, COD emissions reduction targets in five cities---Dalian in Liaoning Province, Ningbo in Zhejiang Province, Xiamen in Fujian Province, Qingdao in Shandong Province and Shenzhen in Guangdong Province---are separately listed paralleling to 30 provinces in Mainland China (Xizang is excluded, because it is uncovered by AESPF). We directly use the targets for these five cities in our analysis.} Even though city's emissions target also could be measured by actual COD emissions proportion of city that accounting for total provincial emissions, we still use the strategy denoted by Equation \ref{eq:${R}_{c}$} in our main empirical analyses due to endogeneity concerns. However, as a robustness test, we rely on the other measurement of ${R}_{c}$ specified as follows,
	\begin{equation} \label{eq:${R}_{c}$ alternative measurement}
	\Delta COD_{c,05-10}=\Delta COD_{p,05-10} \times\dfrac{P_{c,2005}}{\sum_{j=1}^{J}P_{j,2005}}
	\end{equation}
where the second term on the right-hand side of the equation is a measure of the city's proportion of the province's total emissions volume in 2005 based on firm-level emissions data provided by AESPF. Table \ref{tab:robust-RC} in the appendix presents the results.
%The city level COD emissions target inferred from the above method is well grounded not only because volumn of industrial production is one of the critical determinants when assigning emissions targets.\footnote{\textit{see} Approval of the State Council on Emissions Control Plan of Main Pollutants in the 11$^{th}$ Five-Year Plan}, but also because COD emissions are not directly measured but are estimated from production activity by local government officials.%
Figure \ref{fig:rcmap} provides a map of China in which we depict the level of ${R}_{c}$ of all 285 cities in our sample. The darker the color is, the higher the emissions reduction targets are and the stricter the environmental regulations and legal enforcement.
\begin{figure}[htbp]
	\caption{City-Level Regulation Stringency of COD (10 Thousand Tons)}
	\label{fig:rcmap}
	\centering
	\includegraphics[scale=0.1]{rcmap.png}
\end{figure}

There are large variations in pollution intensity across industries of China's manufacturing. Extreme examples are paper production (CIC code 22) as the most heavily polluting industry responsible for 35.16$\%$ of the total COD emissions and recycling and manufacturing of articles for culture, education, and sport activities (CIC code 43 and 24) as the least polluted industry accounting for only 0.013$\%$ of the total COD emissions. Along with the compulsory targets, industries such as paper production, textiles, chemical materials and products, beverage production are identified as "key" industries to lower pollution in China's 11$^{th}$ Five-Year Plan. The response of firms in industries with discrete pollution intensity to stricter environmental regulations is not necessarily the same considering different enforcement shocks that might be exerted.

To investigate the varied reaction of firms across industries with different pollution intensity to stricter environmental regulations before and after 2006, we further run a difference-in-difference-in-differences (DDD) regression based on the following model:
\begin{equation} \label{eq:triple interaction}
y_{it}=\beta_1 \text{R}_{c} \times \text{Post}_{t} +\beta_2 \text{R}_{c} \times \text{Post}_{t} \times \text{Dirty}_{s}  +\beta_3 \text{Dirty}_{s} \times \text{Post}_{t} + \gamma\text{Z}_{c,t-1} +\varphi_i+\varphi_{t} +\epsilon_{it},
\end{equation}
where $y_{it}$ is the log value of firm $i$'s COD, output and pollution intensity at year $t$, respectively. In Equation \ref{eq:triple interaction}, we incorporate a variable ${{Dirty}_{s}}$, that is industry's polluting intensity indicated by each industry's proportion of total COD emissions in all industries in 2005. Table \ref{tab:dirty} in the appendix reports the summary statistics of pollution intensity for all 2-digit manufacturing industries. Definition of other variables assembles those in equation (1). We are interested in co-efficient $\beta_2$, which estimated the heterogeneous effects of environmental regulations on firms' pollution activities across highly polluting industries and cleaner industries. Facilitated by this, we are able to grasp differential effect of environmental regulation stringency on firms' polluting and economic activities across industries with varied polluting level.

Except from the impact of environmental regulations on firms' environmental performances, we are also interested in how firms react to strengthening environmental enforcement, and more importantly, in the underlying forces driving firms to adjust their environmental and economic performance. To reach this multiple goals, we firstly re-define our estimated coefficient $y_{it}$ as legal punishment firms faced, their effluent, water consumption, adoption of pollution control devices, and application for green patent to observe the role of government’s better legal enforcement, firms' recycling practice, pollution abatement facilities, technology progress. We then replace firm $i$'s pollution-related activities with economic indicators including its profits, capital, and labor, as denoted by $y_{it}$.
Accordingly, we rely on firms' pollution data and other firm-level data.

\subsection{Data Description}\label{sec:data}

\subsubsection{Firms' Pollution Data}

The data on firms' pollution emissions come from \textit{Annual Environmental Survey of Polluting Firms} (AESPF) of China.\footnote{Because of the lack of an official name, the dataset was also named the China Environmental Statistics dataset (CESD) \citep*{zhang2018does}, Environmental Statistics Data (ESD) \citep*{wu2017westward}, or the Environmental Survey and Reporting Database (ESRD) \citep*{he2018environmental}.} Established by the Ministry of Ecology and Environment (formerly known as the Ministry of Environmental Protection) from 1980s in a bid to document the state of environmental pollution and abatement in China, AESPF covers rich information on firms' environment-related performance, including emissions of main pollutants (industrial effluent, waste air, COD, $NH_{3}$, $NO_{x}$, $SO_{2}$, smoke and dust, solid waste, noise, etc.), polluting abatement equipment, and energy consumption (usage of freshwater, recycle water, coal, fuel, clean gas, etc.), among others. %Our calculation of total industrial output also relies on this dataset.
Even though gradually normalized during the past 40 years, the scope, frequency, main indicators, reporting method of the environmental survey become largely stable from the starting year of the 10$^{th}$ Five-Year Plan in 2001. For example, as for the scope of the survey, firm will be surveyed as long as one of its pollutants fall into the top 85$\%$ of the total emissions volume of that pollutant at county level.\footnote{During the 9$^{th}$ Five-Year Plan period, industrial polluting sources covered by the survey were limited to state-owned enterprises above the county level and township industrial plants. Even though the scope gradually expanded during the following several Five-Year Plan periods after the 10$^{th}$ Five-Year Plan period, the basic 85$\%$ selection principle of industrial polluting sources remains unchanged.} Those firms are included in a key-point environmental survey list. Once listed, they are obliged to fill uniform statistical statements sent by the environmental authorities to report a wide range of environmental information in last year. Scrutinized and verified by all upper levels of administrative authorities, the data will be confirmed and included in the database.

Like the broadly used Annual Survey of Industrial Firms (ASIF) which provides for basis for macro economy indicators, AESPF is also the sourcing database for calculating macro-level environmental indicators in, for example, China Statistical Yearbook on Environment. In Figure \ref{fig:Data-micromacro1.eps}, we compare the COD emissions volume/industrial effluent aggregated by firm-level data from AESPF and the total volume of industrial COD/industrial effluent from China Statistical Yearbook on Environment. The coincidence between blue/red dotted line and 100$\%$ level provides us more confidence about reliability of AESPF in our empirical analysis. Figure \ref{fig:Data-combine.png} in the appendix compares main air pollutants and pollution abatement facilities between the micro-data and the macro-data. Similar coincidence could be found as for air pollutants and facilities.

\begin{figure}[htbp]
	\caption{Comparison between Micro-Data and Macro-Data on COD and Effluent}
	\label{fig:Data-micromacro1.eps}
	\centering
	\includegraphics[scale=0.8]{Data-micromacro1.eps}
	\fnote{Notes: The macro-data come from China Statistical Yearbook on Environment, and the micro-data come from Annual Environmental Survey of Polluting Firms (AESPF) of China.}
\end{figure}

To investigate the effect of environmental regulations pertaining to COD emissions control in the $11^{th}$ Five-Year Plan from 2006 to 2010, we mainly rely on AESPF data from 2001 to 2010. The cleaned dataset includes 437,253 observations, containing information on 96,378 unique firms.\footnote{When cleaning the data, we only keep information on manufacturing; we exclude firms with missing or zero values for COD and total output in our cleaned dataset.}

\subsubsection{Other Firm-Level Data}

As one of the most comprehensive and widely used Chinese firm-level dataset, the \textit{Annual Survey of Industrial Firms} (ASIF) maintained by the National Bureau of Statistics of China (NBSC) provides the basis for our analysis on firms' economic performance affected by environmental regulations. This data panel covers all state-owned industrial firms and non-state-owned industrial firms with annual sales above 5 million RMB. It contains detailed information on each of those Chinese firms' basic information (name, identification number, registration type, etc.) and information on firms' accounting statement (balance sheet, profit and loss account, and cash flow information). Grounded on firms' name, and then registration number, the merged ASIF and AESPF data contain information for 222,780 observations from 2001 to 2009.\footnote{Although the AESPF data we used are from 2001 to 2010, ASIF data from 2010 include some misreported information and have been generally abandoned by researchers, such as \citet*{fan2018minimum} and \citet*{konig2018imitation}. Therefore, the duration of time in the merged dataset is from 2001 to 2009.} Except from investigating firms' response of production to stricter pollution abatement requirement, we also use the merged dataset to construct robustness checks.

In an effort to find out innovative effect of stricter environmental regulation, the green patent data we use comes from the Chinese Patent Dataset, maintained by the China National Intellectual Property Administration (CNIPA). The dataset records detailed information on each patent applied through CNIPA since 1985, including year, name of the applicants, description of the patent, etc.\footnote{Patent data in China include three categories: invention patents, utility models, and designs. We exclude the third category, "designs", from our patent data, because the common view is that designs lack relevant information about technological innovation.} Firms' performance in green technology induced by stricter environmental regulations is at core of our analysis. We, therefore, follow the IPC classification of environmental-related technologies in \citet*{havsvcivc2015measuring} to identify all the green patent, or more precisely innovation patents and utility models aimed to reduce pollution emissions during the production, from the patent dataset of CNIPA. We further identify firms' green patents specifically on water pollution abatement based on the comparison table in \citet*{havsvcivc2015measuring}. Recognizing that the number of firms owning green patents is limited, patent-related variables are measured by the logarithm of 1 plus the initial number of firms' green patents. We merge the patent data with AESPF on the ground of firms' names.

If contravening the law, firms might be imposed warning, fine, correction with due, or some combination thereof by the government to enforce compliance with regulatory legislation. To construct measures on legal enforcement, we use data on environmental administrative penalties, collected by Institute of Public and Environmental Affairs (IPEA), a well-known Chinese environmental NGO. Administrative authorities are obliged to disclose information on environmental penalties they levy on firms, persons or other organizations through many channels including the internet. The database provides detailed information, from 2004 onward, on environmental penalties, including illegal facts, types of penalties, amount of monetary fines, and implementation of the penalties, faced by firms because of their illegal polluting activities. We merge the environmental penalty data from 2004 to 2010 with the AESPF data based on firms' names in our Section \ref{sec:mechanism}.

In order to see whether firms will relocate to jurisdictions with less stringent environmental regulations, we utilize the State Administration of Industry and Commerce (SAIC) database in China. The SAIC provides complete records of name and domicile of firms, their legal person, registered capital, business scope, shareholders, and what concern us most, their establishment year. Therefore, we are able to trace firm's entrancce in different cities by adding the number of firm firstly established at city level.

We also take advantage of various statistical books, such as China City Statistical Yearbook and China Statistical Yearbook on Environment, and many official documents, for instance, Approval of National Emissions Control Targets of Main Pollutants during the $11^{th}$ Five-Year Plan by the State Council, to obtain the city-level data and industrial COD emissions as well as emissions reduction targets. Table \ref{tab:summarystats} in the appendix reports summary statistics for all the variables.

\section{Main Results}\label{sec:results}

\subsection{Baseline Results} \label{sec:baseline}

To more precisely capture underlying forces that drive firms' environmental performance in response to more stringent environmental regulations, we follow \citet*{martin2011energy} to decompose the within-firm sample as follows:

\begin{equation}\label{eq:$e_{i,t}$}
e_{i,t}=y_{i,t}\times\dfrac{e_{i,t}}{y_{i,t}}
\end{equation}
$e_{i,t}$ is firm $i$'s total pollution at year $t$, which equals to firms' output $y_{i,t}$ multiplied by pollution per unit of output ${e_{i,t}}/{y_{i,t}}$. Taking the log of both sides of equation \ref{eq:$e_{i,t}$}, we have:

\begin{equation}\label{eq:$log(e_{i,t})$}
\Delta log(e_{i,t})=\Delta log(y_{i,t})+\Delta log(\dfrac{e_{i,t}}{y_{i,t}})
\end{equation}
where $\Delta log(e_{i,t})$ refers to changes in firm $i$'s total COD emissions at year $t$. The first term on the right-hand side of Equation \ref{eq:$log(e_{i,t})$} is the "within-firm scale effect", which explains changes of total pollution as firms' overall output increase or decrease. The second term is "within-firm technique effect", accounting for pollution changes brought by changes in firm level pollution intensity, through, for instance, adoption of pollution abatement facilities, introduction of cleaner production process, recycling usage of inputs, among others. We can find that changes of the total emissions can be explained by a within-firm "scale effect" and a "technique effect". The finer within-firm disaggregation is, the more likely we could accurately evaluate the role of different underlying forces in abating water pollution.

Table \ref{tab:baseline} presents the estimation results of Equation \ref{eq:baseline_reg}. All columns include firm fixed effects and year fixed effects. We control for log GDP per capita and log population in all odd columns rather than the odd columns. Columns (1) and (2) present the estimated coefficient for COD; Columns (3) and (4) are the results for output; while the last two columns indicate the results for firms' pollution intensity, that is, COD$/$output. In Column (1), the estimated coefficient is -0.063, which is statistically significant at the 1$\%$ level. This finding suggests that, with the advent of stricter environmental regulations, manufacturing in China is becoming cleaner overall. The extent of emissions reduction, however, varies for cities faced with different environmental regulation stringency denoted by mandatory COD emissions targets set in the 11$^{th}$ years plan period starting from 2006. Compared with cities with lenient environmental regulations, cities in highly regulated cities account for larger pollution reductions. Taking Shanghai and Kunming in Yunan Province as an example. The former ranks first in ${R}_{c}$, that is COD emissions reduction (45 thousand tons) among all large cities, whereas the latter ranks bottom among all large cities in COD emissions reduction target (4.2 thousand tons). Taking Column (2) as our preferred baseline result, in response to stricter emissions control under 11$^{th}$ Five-year Plan, the extent of COD emissions fall in Shanghai is nearly 30$\%$ larger than that in Kunming.\footnote{The discrepancy in the percentage of emissions reduction between these two cities brought on by stricter emissions control from 2006 is calculated as $-0.073*(45-4.2)=-0.29784$.}

An important assumption of our DID identification strategy is that the different over-time changes in pollution activities across firms at cities with different level of environmental regulation stringency are solely caused by the laying out of reduction target set in the 11$^{th}$ Five-Year Plan, rather than by any pre-existing differential time trends across firms. To test this assumption, we replace the interaction between environmental regulation stringency and the post dummy in Equation \ref{eq:baseline_reg} with the sum of the interaction terms between environmental regulation stringency and all the year dummies. Figure \ref{fig:dynamicscod} in the appendix plots the estimated yearly effects of environmental regulation stringency $R_{c}$ on firms' COD emissions. We observe no significant pre-trend before 2005 but a break in 2005.

When looking at Columns (4) and (6) of Table \ref{tab:baseline}, we perceive that, among 7.3$\%$ of the variation in pollution reduction across cities with different environmental regulation stringency after 2006, 2.2$\%$ of them can be explained by within-firm "scale effect", that is, drop in firms' total output, whereas 5.1$\%$ of those can be explained by within-firm "technique effect", including but not limited to adoption of pollution abatement facilities, introduction of cleaner production process and recycling use of inputs. A back-of-the-envelope calculation reveals that, within-firm "technique effect" is the predominant determinant, contributing up to 70$\%$ (-0.051/-0.073) to the effect of environmental regulation stringency on firms' pollution reduction, whereas the within-firm "scale effect" accounts for the other 30$\%$ (-0.022/-0.073).\footnote{It is noteworthy that the weight of the "technique effect" in accounting for overall changes in manufacturing emissions in our cross-industry decomposition in \ref{sec:decomposition} is larger than it is here. This finding implies that, from the viewpoint of individual firms, adjustments in output will be an easier way to trade off between an output of production and costs in meeting stricter emissions control requirements. From the viewpoint of industries, however, shifts among industries happen more.} Our benchmark results imply that, when the shock of stringent environmental regulations comes, firms located in cities with more stringent environmental regulations are stimulated to reduce more pollutants largely through technique progress, compared with their counterparts in cities with lenient environmental regulations. In addition, Table \ref{tab:robust-RC} in the appendix shows that our baseline results are robust to alternative method of calculating environmental regulation stringency.
\input{baseline}

We further test the impact of environmental regulation stringency on other pollutants in Table \ref{tab:otherpollutant}. All columns include firm fixed effects and year fixed effects. We control for log GDP per capita and log population in all odd columns rather than the odd columns. First, we execute a test about effect of environmental regulation stringency on firms' SO$_{2}$ emissions, another one of the two pollutants regulated by top-down mandatory reduction targets in the 11$^{th}$ Five-Year Plan. In Columns (1) and (2) of Table \ref{tab:otherpollutant}, we also find that, upon advent of the pollution reduction commitment period from 2006, firms' SO$_{2}$ emissions decrease, and the negative effect with similar magnitude with those in the baseline is, more notably, declared for cities with more stringent environmental regulations. Second, we study the effects of environmental regulations on two other pollutants NH$_{3}$-N and smoke and dust before and after 2006. Since these two pollutants are not "critical pollutant" in the 11$^{th}$ Five-Year Plan, the strictness in environmental regulations should have little effect on emissions if our result in Table \ref{tab:baseline} are not driven by confounding factors.   In contrast, if the results were entirely driven by confounding factors, the confounding factors should also apply to pollutants not regulated by obligatory emissions reduction targets. As can be told from Columns (3) to (6), we do not find emissions of NH$_{3}$-N and smoke and dust to be significantly affected by environmental regulation stringency. Therefore, Columns (3) to (6), which provide a placebo test, rule out strong confounding factors as being responsible for the relationship between environmental regulation stringency and pollution emissions.

\input{otherpollutant}

\subsection{Heterogeneous Effects by Industry}\label{sec:heterogeneous}

As our third stylized fact in Section \ref{sec:facts} reveals, the effect of environmental regulation stringency on firms' COD emissions varied across industries with different polluting intensity. Industrial polluting intensity may positively reinforce the effect of environmental regulation stringency on firms' pollution reduction. To testify the heterogeneous effect, we estimate Equation \ref{eq:triple interaction}, and Table \ref{tab:heterogeneous} presents the results.

In Table \ref{tab:heterogeneous}, the dependent variables are COD (see Columns (1) and (2)), output (see Columns (3) and (4)) and firms' pollution intensity (see Columns (5) and (6)), respectively. We control for firm fixed effect and year fixed effect in all columns. By adding city-level variables in Columns (2), (4), and (6), the results are not substantively different from those in the odd columns. Combining the estimates of ${R}_{c} \times{Post}_{t}\times{Dirty}_{s}$ as well as ${R}_{c} \times{Post}_{t}$ which are both negative in all columns, we find that, firms' pollution emissions decrease in all industries. The extent of emissions reduction, however, varies across industries. Firms that belong to heavily polluting industries located in cities with more stringent environmental regulations cut down much more on pollutant use after 2006, compared with their intra-city counterparts in less-polluting industries. Here is a back-of-the-envelope calculation on the ground of Column (2). Let's take the two noticeable industries---manufacturing of paper and paper products and manufacturing of articles for culture, education, and sport activities---in Shanghai and Kunming again as examples.\footnote{Among all two-digit industries, the paper and paper products industry reports the highest pollution intensity with a ${Dirty}_{s}$ of 0.35164, whereas the manufacturing of articles for culture, education, and sport activities has the smallest industrial pollution intensity at 0.00013.} Compared with paper industry firms in Kunming, firms in the same industry in Shanghai reduce 69.4$\%$ more COD emissions upon the shock of strengthening emissions reduction enforcement ((-0.339$\times$0.35164-0.051)$\times$(4.5-0.42)=-0.694). Firms in the latter industry in Shanghai reduce, however, only 20.8$\%$ more COD in relation to their counterparts in Kunming ((-0.339$\times$0.00013-0.051)$\times$(4.5-0.42)=-0.208). Much more sensibility of firms in heavily polluting industries to environmental regulations after 2006 lends solid evidence to our stylized facts about the inter-industry allocation.

As a robustness check, we further use a dummy variable to measure industrial pollution intensity. The variable equals to one for heavily polluting industries, and equals to zero for lightly polluting industries. As shown in Table \ref{tab:robust-dirty}, the results remain similar. %We rank the COD emissions of industries from top to bottom by proportion of each industry's COD emissions, the industries who aggregately contribute to 50\% of total industrial emissions are definded as heavily polluting industries. 
We also conduct robustness checks by using information on above-scale firms from ASIF data. Our results are robust to the sample adjustment, as shown in Table \ref{tab:robust-ASIF} in the appendix.

 \input{heterogeneous}

\section{Mechanism}\label{sec:mechanism}

Manufacturers in China is "greening", as observed in the previous analysis. It thus becomes natural to ask: what are the inherent mechanisms playing a role when firms adapting to a new era of stringent environmental regulation? When answering this question, it is obviously hasty to pin the forces down until activities at the firm level are deliberately examined. Benefited from various firm-level data, we are able to execute a broad spectrum of tests on roles of recycling practice (recycling of wastewater), adoption of pollution abatement facilities (wastewater treatment), and firms' technology progress (green patent), among others.

\subsection{Environmental Penalties}\label{sec:environmental penalty}

Despite that targets assigned to each level of governments and polluters are obligatory, effectiveness of target control heavily relies on the strength of daily legal enforcement, such as environmental penalties levied by authorities on firms, which is also the most frequently used regulatory tools in China. As an important prerequisite, we need to reassure that legal enforcement on emissions target control has been tightened which could be explicated by changes of probability that firms might be punished before and after 2006.

To do so, we construct two variables, $Polluter\ Penalty\ Dum$ and $Polluter\ Penalty\ Num$. The former one is a dummy variable denoting whether a firm was punished for violations of emissions limitations and other legal obligations. To be specific, $Polluter\ Penalty\ Dum$ equals to 1 in the year when the firm was penalized and otherwise equals 0. The latter one, nevertheless, refers to frequency with which the firm was penalized in a certain year. $Polluter\ Penalty\ Num$ was numerated with the sequence of 1,2,3...according to times firm was punished.

As shown in Table \ref{tab:IPE}, after launching year 2006 of the 11$^{th}$ Five-Year Plan, the strengthened environmental regulations significantly increase the probability firms might be punished. The increasing of probability is much more declared for firms located in cities with stricter environmental regulations and in industries with higher pollution intensity.

\input{IPE}

\subsection{Further Decomposing Within-Firm Pollution Activities}\label{sec:further decomposition}

First to come, we execute a finer within-firm decomposition based on Section \ref{sec:baseline} by introducing another pollutant---effluent. Effluent is an important water pollution parameter, because it not only reflects the total volume of industrial water entering into the natural environment, but also provides us a medium to assess the real determinants of firms' environmental performance. Therefore, firms' effluent discharge can be further decomposed into

\begin{equation}\label{eq:log(f_{i,t})}
\Delta log(f_{i,t})=\Delta log(y_{i,t})+\Delta log(\dfrac{f_{i,t}}{y_{i,t}}),
\end{equation}
where $\Delta log(f_{i,t})$ is change of effluent discharge of firm $i$ at year $t$. Changes in firms' output is expressed by $\Delta log(y_{i,t})$. The second term on the right-hand side of the equation refers to changes of firm $i$'s pollution intensity, that is effluent discharge per unit of output.

We repeat the regression of our preferred baseline specification and the triple interaction with industrial polluting intensity for each of above two components. Columns (1) and (2) of Table \ref{tab:Effluent} present the results for firms' effluent discharge. Columns (3) and (4) report the results for firms' pollution intensities. In all columns, we add city-level controls, firm fixed effect, and year fixed effect. As shown in Column (1) of Table \ref{tab:Effluent}, the coefficient on firms' effluent is negative and statistically significant, which implies that firms' discharge of effluent declines adhering to similar patterns as COD emissions after 2006. In other words, the extent of effluent discharge is negatively associated with stringency of environmental regulations. After including the triple interaction, the point estimates of $R_{c}\times Post_{t}$ and $R_{c}\times Post_{t}\times Dirty_{s}$ in Column (2) are both negative. It indicates that firms' effluent declines in all industries after strengthening the environmental regulations on target polluting reduction in 2006, the effect is, however, much stronger for heavily polluting industries. We find similar effects on firms' effluent discharge per unit of output as shown in Columns (3) and (4). That is to say, decreased industrial wastewater discharge (effluent) is the main reason for firms' emissions reduction.

\input{Effluent}	

\subsection{Water-Related Energy Consumption}\label{sec:energy}

Then how firms' cutting-down of effluent has been realized? We now turn to examine the adjustment of firms' total water consumption (industrial water) and freshwater consumption (freshwater) per unit of output affected by environmental regulations. Recycling use of industrial water is a feasible approach to "kill one bird with two stones" with lower cost for firms to save energy and reduce pollution emissions. We thus expect that the beneficial effect of environmental regulations on firms' emissions reduction through water recycle should be stronger for firms in cities with stricter environmental regulations and those in industries with higher level of pollution intensity. The negative coefficients in Columns (1) and (3) of Table \ref{tab:energy} prove our speculation that, relative to cities where regulations are more weaken, firms in cities where environmental regulations are more stringent consume less industrial water and freshwater after 2006. As we can tell from Columns (2) and (4) of Table \ref{tab:energy}, for firms in more heavily polluting industries, the decline of water consumption are much sharper, among which the fall of freshwater consumption is especially spectacular.

 \input{energy}

\subsection{Adoption of Pollution Control Devices}\label{sec:facilities}

The adoption of pollution control devices refers to devices installed and operated to eliminate emissions of pollutants in effluent entering natural waterways. To meet mandatory emissions cap and reduction target on COD, firms, put aside cutting down output and recycling water, will choose to install more pollution control devices and to expand the treatment abilities. We thus examine the effect of environmental regulation stringency on firms' pollution abatement. Column (1) of Table \ref{tab:facilities} shows the estimation results for adoption of pollution control devices divided by firms effluents. We find that the interaction effects between environmental regulation stringency and devices per unit of effluent after 2006 is positive at 1$\%$ significant level. By summing up the triple interaction among environmental regulation stringency, post 2006 dummy, and industrial pollution intensity, we observe positive and statistically significant results in Column (2) Table \ref{tab:facilities}. The estimates on ability of firms' pollution control devices per unit of effluent reveals similar positive results in Columns (3) and (4) with those in Columns (1) and (2). After introducing stringent environmental regulations in 2006, firms adopt more environmental abatement devices and expand the treatment capacity. The increasing of control devices and their treatment capacity is more claimed for firms in cities with stringent environmental regulations than their counterparts in cities with weaken environmental regulations after 2006. Relative to firms in cleaner industries, firms in dirtier industries tend to be more progressive in expanding pollution treatment abilities.

 \input{facilities}

\subsection{Patent}

Even though pollution abatement devices always involve technological renews, it is actually somewhat end-of-pipe solutions because the devices are usually purchased from the market, and thus barely need R$\&$D of firms. In this subsection, we turn to examine firms’ performance in terms of "bright green" induced by stricter environmental regulations by using disaggregated firm patent data provided by National Intellectual Property Administration of China.

However, we find little evidence that severity of environmental regulations is associated with increase of green patent and water-related green patent applications, regardless of what industry firms belongs to. The weak negative results in Columns (1) to (4) are even indicative of the possibility that environmental regulations impede firms’ environment-related innovation. Thus, we are able to infer that emissions target control during the 11$^{th}$ Five-Year plan failed to stimulate firms regulated to adopt much more effective technology to eliminate more pollutants sourcing from the production process, maybe because other lower-cost and end-of-pipe countermeasures, such as pollution abatement facilities, is sufficient to meet the target. Moreover, when we further divide the two variables in Table \ref{tab:patent} into green innovation patents, green utility models, water-related green innovation patents, and water-related utility models according to general classification of patent in China. The results in Table \ref{tab:robust-patent} in the appendix are similar to those in Table \ref{tab:patent}.

\input{patent}

\section{Firms' Other Economic Performance}\label{sec:performance}

The preceding Section \ref{sec:results} verify the positive effect of environmental regulations on firms' pollution reduction, especially for those in heavily polluted industries. The significantly reduced pollution, output and pollution per unit are ought to influence firms' economic performance. To this end, we accordingly conduct a test in this section to further assess the impact of environmental regulations on firms' economic performance, including profits, capital, labor, and market share.

Basically, Columns (1), (3), (5) and (7) of Table \ref{tab:performance} show that, with firm fixed effect, year fixed effect and city-level variables controlled for, severer environmental regulations is associated with sharp decline in firms' profits, capital, labor and market share. The extent of firms' worsened performance in economics, nevertheless, is different for firms located in cities with varied environmental regulation stringency. Compared with firms in cities with lenient environmental regulations, firms in highly regulated cities experience larger decreases in profits, capital, labor and market share as a result of environmental regulations. By incorporating the triple interaction among environmental regulation stringency, industrial polluting intensity and post year dummy, the consistently negative estimates of $R_{c}\times Post_{t}$ and $R_{c}\times Post_{t}\times Dirty_{s}$ in Columns (2), (4), (6) and (7) of Table \ref{tab:performance} is indicative of decrease in firms' profits, capital, labor and market share in all industries. Also, as for firms respectively located in tightly and loosely regulated cities, heavily polluting industry firms experience much more decline in profits, capital, labor and market share, compared with their counterparts in cleaner industry. For example, regardless of discrepancy in industrial polluting intensity, the average fell of firms' profits, capital, labor and market share in Shanghai is 67.9$\%$, 23.5$\%$, 3.9$\%$ and 1.2$\%$ larger than those in Kunming of Yunan Province, respectively; When taking into account of industrial pollution intensity, compared with firms in paper industry in Kunming, profits, capital, labor and market share of firms in the same industry in Shanghai experience 101.6$\%$, 51.4$\%$, 21$\%$ and 1.5$\%$ more plunge; firms in the manufacturing of articles for culture, education, and sport activities, the cleanest industry concerning COD emissions, in Shanghai encounter only 60.2$\%$, 17.1$\%$, 0.2$\%$ and 1.1$\%$ more decline in their profits, capital, labor and market share.


\input{performance}

A decrease in output, input and market share of firms might shed light on potential adjustment of firm relocation, we therefore further testify the impact of environmental regulation on firm entry. Two variables \textit{Entry Num} and \textit{Entry Capital}, which denote log value of firms' number newly registered in a city and log value of their total registered capital, respectively, are introduced here.\footnote{In short of information on industries of registered firms in SAIC, we only include the interaction term ${R}_{c} \times \text{Post}_{t}$ in our regression.} The significantly negative estimates in Columns (1) and (2) in Table \ref{tab:entry} show that, entrance of firms in cities with stringent environmental regulation plunged sharply compared with those in loosely regulated cities after 2006. The shrink of overall registered capital in highly regulated cities presents similar pattern, as shown in Columns (3) to (4). Their escape signifies, to a certain extent, an "internal" variant of the pollution havens hypothesis.

\input{entry}

\section{Conclusion}  \label{sec:conclusion}

This paper examines the effect of environmental regulations on firms' COD emissions reductions. We find that with the advent of stricter environmental regulations, represented by the differential emissions reduction targets set up in the Chinese 11$^{th}$ Five-Year Plan after 2006, manufacturers have emitted less COD. More stringent environmental regulations faced by firms is positively associated with a greater probability of reducing COD emissions after 2006. Also, firms belonging to heavily polluting industries have since cut down their pollutant use by much more when compared with their counterparts in less-polluting industries. We find no such effects of the policy on other non-targeted pollutants, such as $NH_{3}$ and smoke and dust. Our analyses suggest that stricter environmental regulations have induced firms to pay more efforts to COD emissions-related issues. By constructing a comprehensive dataset, we execute a series of tests to determine the underlying mechanisms affecting firms' reactions to stringent environmental regulations. With the stricter target control system in place, firms face a higher probability of receiving administrative penalties and are more likely to discharge less effluent, to consume less industrial water by recycling water, and to adopt devices that control pollution as well as expand their current pollution treatment abilities. However, we find no evidence of an increase in green patents and water-related green patent applications, leading us to believe that firms are, nevertheless, reluctant to increase environment-related innovation.

Our research has three important implications. First, as environmental regulations become more tightened, the amount of firms' emissions will reduce. Encountered with concrete emissions reduction targets, firms are consciously trading off between production arrangements and COD emissions. Overall, manufacturing firms in China are becoming green. Second, the industrial structure in China is also becoming cleaner due to the expansion of cleaning industries, while polluting industries are shrinking. As firms in heavily polluting industries are more responsive to environmental regulations, their sharply declining output and the pollution intensities of their production provide more evidence of the "composition effect". Third, the effective of pollution reduction though it is, environmental regulations by setting clear reduction targets during the 11$^{th}$ Five-Year Plan still fail to stimulate firms to adopt effective technology to eliminate more pollutants sourcing from the production process, maybe because other lower-cost countermeasures, such as adoption of pollution abatement facilities, is sufficient to meet the target. As for regulators, how to find out appropriate regulatory path to stimulate firms to shift from adoption of end-of-pipe treatment technology to innovation is fundamental tasks. Despite the common acceptance on superiority of market-based over command-and-control instruments in terms of environmental regulatory induced innovation \citep*{ambec2013porter}, whether it is the type of regulation matters, or those who can exerts continuous pressure on firms will be effective ones regardless of their types still need further investigation.

\clearpage

\setcounter{table}{0}
\setcounter{figure}{0}
\renewcommand{\thetable}{A\arabic{table}}
\renewcommand{\thefigure}{A\arabic{figure}}
\renewcommand{\thesubsection}{Appendix \Alph{subsection}}
\renewcommand{\thesubsubsection}{Appendix \Alph{subsection}.\arabic{subsubsection}}
\clearpage
\newpage

\bibliographystyle{aernobold}
\bibliography{ReferenceFile}
\clearpage
\newpage
\section*{Appendix}

\subsection{Dynamic Trend}

\begin{figure}[htbp]
	\caption{Dynamics of Chemical Oxygen Demand Emissions}
	\label{fig:dynamicscod}
	\centering
	\includegraphics[scale=0.8]{Dynamics-lnhxxy.eps}
	\fnote{Notes: This figure plots the estimated coefficients of environmental regulation stringency and year dummy variables (controlling for the log per capita city GDP and log city population, and year and firm fixed effects) and their 90\% confidence intervals. The reference year is 2001.}
\end{figure}

\clearpage
\newpage

\begin{figure}[htbp]
	\caption{Comparison between Macro- and Micro- Data on Main Air Pollutants and Control Facilities}
	\label{fig:Data-combine.png}
	\centering
	\includegraphics[scale=0.1]{Data-combine.png}
	\fnote{Notes: The macro-data come from China Statistical Yearbook on Environment, and the micro-data come from Annual Environmental Survey of Polluting Firms (AESPF) of China.}
\end{figure}

\clearpage
\newpage

\subsection{Summary Statistics}
\input{summarystats}

\clearpage
\newpage

\subsection{Summary of Pollution Intensity}
\input{dirty}

\clearpage
\newpage

\subsection{Using Different Measure of COD Regulation}

\input{robust-RC}
\clearpage
\newpage

\subsection{Using Different Measure of Industry-Level COD Intensity}
\input{robust-dirty}
\clearpage
\newpage

\subsection{Patent Classification}
\input{robust-patent}


\clearpage
\newpage


\subsection{Using ASIF data}
\input{robust-ASIF}

\clearpage
\newpage

\end{document}
